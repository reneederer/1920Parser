\subsection{Pflichtenheft (Auszug)}
\label{app:Pflichtenheft}
Die Anwendung muss folgende Anforderungen erfüllen: 
\begin{enumerate}[itemsep=0em,partopsep=0em,parsep=0em,topsep=0em]
\item Muss-Kriterien:
	\begin{enumerate}
	\item Datenstrom und Schema müssen frei angegeben werden können.
	\item Das Programm muss den Datenstrom anhand des Schemas zergliedert anzeigen.
	\end{enumerate}
\item Soll-Kriterien:
	\begin{enumerate}
	\item Benutzer sollen ein eingegebenes Schemas speichern können.
	\item Benutzer sollen gespeicherte Schemas laden können.
	\end{enumerate}
\item Kann-Kriterien:
	\begin{enumerate}
    \item Anhand des Transaktionscodes im Datenstrom soll das richtige Schema - falls vorhanden - automatisch ausgewählt werden.
    \item Der Benutzer soll ein Kriterium angeben können, anhand dessen die passenden Datenströme aus der Logdatei ausgewählt werden.
    \item Wenn das Länge-Feld eine Angabe zur Anzahl der Nachkommastellen hat, soll das Programm in der Ergebnis-Anzeige an der richtigen Stelle ein Komma einfügen.

    \item Variablenzeilen sollen ausgeblendet werden können.
        \end{enumerate}
\item Ausschluss-Kriterien:
	\begin{enumerate}
    \item Das Programm soll Schema und Datenstrom nicht auf Korrektheit oder Plausibilität prüfen
(z. B. doppelte Variablennamen, Redefines ohne Vorgänger, ungültige Stufennummern, Bytelänge 0, Char mit Nachkommastellen).
    \end{enumerate}
\end{enumerate}
