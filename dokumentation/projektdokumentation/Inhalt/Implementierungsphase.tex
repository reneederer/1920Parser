% !TEX root = ../Projektdokumentation.tex
\section{Implementierungsphase - Zergliedern des Datenstroms} 
\label{sec:ImplementierungsphaseDatenstromZergliedern}
Die Implementierung begann mit dem Anlegen eines neuen \ac{SVN}-Repositories mit der vorgegebenen Verzeichnisstruktur (branch, tag, trunk) und dem Erstellen einer neuen Solution in Microsoft Visual Studio Professional 2010 mit einem C\# Winforms Projekt. Der Solution wurde ein NUnit-Testprojekt für die Unit-Tests hinzugefügt.

\subsection{Implementierung der Benutzeroberfläche}
\label{sec:ImplementierungBenutzeroberflaeche}
Visual Studio erstellte mit Neuanlage des Winform-Projekts automatisch eine Form-Klasse. Mit dem GUI-Editor von Winforms wurden dieser Klasse drei Textfelder hinzugefügt, jeweils eines für Datenstrom, Schema und Ergebnis. Es wurde noch ein Button zum Starten der Verarbeitung eingefügt. Dieser wurde im Laufe der Implementierung  entfernt und die Verarbeitung bei jeder Eingabe in Schema- oder Datenstrom-Textfeld gestartet.

\subsection{Implementierung der Geschäftslogik}
\label{sec:ImplementierungGeschaeftslogik}
\subsubsection{Parsen des Schemas in eine Baumstruktur}
\label{sec:ParsenSchema}
\paragraph{Parsen einer Variablenzeile zu einer AbstractNode}

Aus jeder Variablenzeile im Schema wird eine Group- oder ValueNode erstellt. Um dies Umzusetzen wurde in der Klasse Schema die Methode ParseLine geschrieben. Die Methode splittet die Angaben einer Variablenzeile mit Hilfe einer \ac{Regex} in ihre Felder auf. Die \ac{Regex} wurde auf \href{https://regex101.com}{https://regex101.com} angepasst und getestet, bis sie richtig funktionierte.

\paragraph{Erstellen einer Baumstruktur aus den Variablenzeilen}
Dies war die schwierigste Aufgabe des Projektes. Die Schemazeilen mussten durchlaufen werden, aus ihnen AbstractNodes erstellt und diese in eine Baumstruktur gebracht werden.
Schwierig war, dass mit jeder neuen  Zeile ein Kindknoten, ein Geschwisterknoten, oder ein Geschwisterknoten eines früheren Vorfahrens auftauchen konnte. In jedem dieser Fälle konnte unterhalb des Knotens erneut eine Hierarchie von Knoten sein. Es musste diese Hierarchie, falls vorhanden, erstellt werden, danach musste die Verarbeitung mit der richtigen Schemazeile und dem richtigen Elternknoten weitergehen.
Schwierig war insbesondere auch, dass eine Schemazeile zu mehreren AbstractNodes werden konnte, abhängig von ihrer Wiederholzahl. In dieser Hinsicht waren besonders GroupNodes problematisch, da sich bei GroupNodes mit der Wiederholzahl der gesamte Baum unterhalb von ihnen mit-wiederholt. Außerdem konnten Wiederholzahlen auch bei mehreren Vorfahren-Knoten vorkommen, so dass sich die Wiederholungen verschachtelten.

Nach vielen vergeblichen Versuchen, mit Rekursion und verschachtelten Schleifen (zum Durchlaufen der Schemazeilen und Erstellen der Wiederholungen von Nodes) zu arbeiten, führten schließlich zwei Ideen zur Lösung des Problems. Die erste war, nicht mit Rekursion zu arbeiten sondern explizit mit einem Stack. Damit wurde es einfacher den Verlauf des Algorithmus  nachzufolgen und vorherzusehen. Die zweite und entscheidende Idee war, Nodes mit Wiederholzahl nicht sofort zu kopieren, sondern sie erst einmal fertigzustellen und danach zu kopieren. Hierzu wurden die Node-Klassen um die Methode CreateCopyWithIndex(: int) erweitert.

Etwas ärgerlich war, dass die Methode zuerst nur so aussah, als würde sie funktionieren. Es waren nur flache Kopien der Nodes erzeugt worden, ohne Kopien der Kindknoten zu erzeugen. Dadurch erhielten alle Nodes von Wiederholgruppen beim Zuweisen die gleichen Werte. Der Fehler war durch Unittests nicht entdeckt worden. Zur Behebung des Fehlers musste CreateCopyWithIndex() seine Kind-Nodes rekursiv kopieren.


\subsubsection{Grundschema der Methoden von AbstractNode}
\label{sec:RekursiveMethoden}
Group- und ValueNode erben von AbstractNode Methoden. Die Implementierung all dieser Methoden ähnelt sich, sie folgt für Group- und ValueNodes immer diesem Prinzip:

ValueNode macht eine Aktion und gibt danach einen Wert zurück.

GroupNode macht eine Aktion und ruft danach für jedes seiner Kinder die gleiche Methode erneut auf (Rekursion). Aus den Rückgabewerten der Kinder wird ein Wert akkumuliert und dieser zurückgegeben.

Die von einer GroupNode angestoßenen Rekursionen enden bei ValueNodes und GroupNodes ohne Kindknoten, die Abbruchbedingung der Rekursion ist polymorph.

Es ist kein Zufall, dass alle Methoden rekursiv sind, denn auch die Klassenstruktur ist rekursiv (GroupNodes können auf GroupNodes verweisen).
\Zwischenstand{Implementierungsphase}{Implementierung}











