% !TEX root = ../Projektdokumentation.tex
\section{Projektplanung} 
\label{sec:Projektplanung}


\subsection{Projektphasen}
\label{sec:Projektphasen}
Das Projekt findet im Zeitraum vom 11.04.2016 - 15.05.2016 statt.
Genaue Zeitplanung

\paragraph{Beispiel}
Tabelle~\ref{tab:Zeitplanung} zeigt ein Beispiel für eine grobe Zeitplanung.
\tabelle{Zeitplanung}{tab:Zeitplanung}{ZeitplanungKurz}\\
Eine detailliertere Zeitplanung findet sich im \Anhang{app:Zeitplanung}.


\subsection{Abweichungen vom Projektantrag}
\label{sec:AbweichungenProjektantrag}
Die Analyse des Aufbaus der 1920Schema-Dateien wird in der Analysephase durchgeführt (statt in der Entwurfsphase), damit die Anforderungen vor Erstellung des Lastenheftes klar definiert werden können.


\subsection{Ressourcenplanung}
\label{sec:Ressourcenplanung}
Windows 7, Visual Studio 2010, Microsoft Visio, TexMaker, OpenText HostExplorer,  Büro mit PC mit Verbindung zum Mainframe
PC, Büro, Projektbetreuer, Strom, Kaffee
\begin{itemize}

	\item Detaillierte Planung der benötigten Ressourcen (Hard-/Software, Räumlichkeiten \usw).
	\item \Ggfs sind auch personelle Ressourcen einzuplanen (\zB unterstützende Mitarbeiter).
	\item Hinweis: Häufig werden hier Ressourcen vergessen, die als selbstverständlich angesehen werden (\zB PC, Büro). 
\end{itemize}


\subsection{Entwicklungsprozess}
\label{sec:Entwicklungsprozess}
Für die Vorgehensweise nach dem Wasserfallmodell spricht, dass die Anforderungen an das Programm sehr klar umrissen werden können. Der Aufbau der 1920Schemas wird sich fast sicher während des Projektes nicht ändern.

Das Vorgehen nach einer agilen Methodik, insbesondere \ac{TDD}, verspricht dagegen eine erleichterte Implementierung, weniger Debugging, Sicherstellung dass benötigte Funktionalität auch macht, was sie soll. Daher wurde das Projekt nach dem agilen Vorgehensmodell durchgeführt.






