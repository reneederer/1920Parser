% !TEX root = ../Projektdokumentation.tex
\section{Projektplanung}
\label{sec:Projektplanung}


\subsection{Projektphasen}
\label{sec:Projektphasen}
Für das Projekt standen 70 Stunden zur Verfügung. Es fand im Zeitraum vom 11.04.2016 - 15.05.2016 statt.

\paragraph{Beispiel}
Tabelle~\ref{tab:Zeitplanung} zeigt die grobe Zeitplanung für das Projekt.
\tabelle{Zeitplanung}{tab:Zeitplanung}{ZeitplanungKurz}\\
Eine detailliertere Zeitplanung findet sich im \Anhang{app:Zeitplanung}.

\subsection{Ressourcenplanung}
\label{sec:Ressourcenplanung}
Zur Projektdurchführung werden folgende Ressourcen eingesetzt:
Windows 7, Visual Studio Professional 2010, Microsoft Visio 2010, TexMaker, XMLSpy 2007, OpenText HostExplorer, Büro mit PC mit Verbindung zum Mainframe und Verbindung zum  Internet, Strom, Projektbetreuer


\subsection{Entwicklungsprozess}
\label{sec:Entwicklungsprozess}
Es wurde ein agiler Entwicklungsprozess angewendet, der an Extreme Programming angelehnt war. Der im Projekt angewandte Prozess beinhaltete die Extreme Programming-Praktiken, testgetriebene Entwicklung, häufige Kundeneinbeziehung, häufiges Refaktorisieren, kurze Iterationen und einfaches Design. 






