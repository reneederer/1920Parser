% !TEX root = ../Projektdokumentation.tex
\section{Analysephase} 
\label{sec:Analysephase}

\subsection{Ist-Analyse} 
\label{sec:IstAnalyse}
Die Ist-Analyse fand als Befragung von Herrn Kemmer statt. Herr Kemmer ist sowohl Projektbetreuer als auch potentieller Benutzer und Auftraggeber des Projektes. Der Autor bereitete eine Liste mit Fragen vor.

Bei der Befragung stellte sich heraus, dass die Entwickler der Abteilung Lager bei Kundenreklamationen und Programmänderungen oft Logdateien analysieren müssen.welchen Umständen genau das Programm benötigt wird, woher die Schemas stamim Datenstrom verbringen men, wofür sie verwendet werden, wie oft sie verwendet werden, von wie vielen, wie hoch er die Zeitersparnis schätzt, wie die Entwickler bisher arbeiten ohne das Programm. Der Autor erfragte genau die Anforderungen und ordnete sie in Muss-, Soll-, und Kann-Kriterien.


\subsection{Wirtschaftlichkeitsanalyse}
\label{sec:Wirtschaftlichkeitsanalyse}
Das Programm soll den Entwicklern das bisher notwendige, fehleranfällige Abzählen von Zeichen in Schema und Datenstrom abnehmen. Das Projekt verspricht dadurch nicht nur, den Entwicklern Zeit zu sparen, sondern auch Zählfehler wirkungsvoll zu verhindern.

\subsubsection{\gqq{Make or Buy}-Entscheidung}
\label{sec:MakeOrBuyEntscheidung}
Die Anforderungen sind so speziell, dass fast auszuschließen ist, dass es außerhalb von Phoenix ein Programm gibt, das die Anforderungen erfüllt. Zu bemerken ist aber, dass der Mainframe von Phoenix mit Schemadateien arbeitet und vielleicht Ähnliches leistet, was das zu erstellende Programm leisten soll. Der Autor fragte deshalb bei seinem Auftraggeber nach, ob der Mainframe-Quelltext anzusehen ist. Herr Kemmer antwortete, dass das zwar möglich, aber sehr aufwändig sei. Da nicht sicher war, dass die Ansicht des Quelltextes die Programmerstellung erleichtern würde, wurde auf eine weitere Verfolgung dieser Idee verzichtet. Es wurde entschieden, das Programm selbst neu zu schreiben.

\subsubsection{Projektkosten}
\label{sec:Projektkosten}
Im Rahmen des Projektes fallen Kosten für Entwicklung und Abnahmetest an.

\paragraph{Beispielrechnung (verkürzt)}
Die Kosten für die Durchführung des Projekts setzen sich sowohl aus Personal-, als auch aus Ressourcenkosten zusammen.
Der Projektersteller ist Umschüler und erhält deshalb von seinem Ausbildungsbetrieb keine Vergütung.

\begin{eqnarray}
7,7 \mbox{ h/Tag} \cdot 220 \mbox{ Tage/Jahr} = 1694 \mbox{ h/Jahr}\\
\eur{0}\mbox{/Monat} \cdot 13,3 \mbox{ Monate/Jahr} = \eur{0} \mbox{/Jahr}\\
\frac{\eur{0} \mbox{/Jahr}}{1694 \mbox{ h/Jahr}} = \eur{0,00}\mbox{/h}
\end{eqnarray}

Dadurch ergibt sich also ein Stundenlohn von \eur{0,00}
Die Durchführungszeit des Projekts beträgt 70 Stunden. Für die Nutzung von Ressourcen\footnote{Räumlichkeiten, Arbeitsplatzrechner etc.} wird 
ein pauschaler Stundensatz von \eur{12} angenommen. Für die anderen Mitarbeiter wird pauschal ein Stundenlohn von \eur{23} angenommen. 
Eine Aufstellung der Kosten befindet sich in Tabelle~\ref{tab:Kostenaufstellung} und sie betragen insgesamt \eur{1015,00}.
\tabelle{Kostenaufstellung}{tab:Kostenaufstellung}{Kostenaufstellung.tex}


\subsubsection{Amortisationsdauer}
\label{sec:Amortisationsdauer}
Es wird davon fausgegangen, dass ausschließlich die Entwickler der Abteilung Lager das Programm nutzen werden. Nach Einschätzung von Herrn Kemmer arbeitet die Hälfte der 20 Entwickler regelmäßig mit Schemadateien. Er schätzte weiter, dass das Programm jedem täglich 10 Minuten einsparen kann.
Bei einer Einsparung von 10 Minuten am Tag für 10 Entwickler an 220 Arbeitstagen im Jahr ergibt sich eine gesamte Zeiteinsparung von 
\begin{eqnarray}
10 \cdot 220 \mbox{ Tage/Jahr} \cdot 10 \mbox{ min/Tag} = 22000 \mbox{ min/Jahr} \approx 366,67 \mbox{ h/Jahr} 
\end{eqnarray}

Dadurch ergibt sich eine jährliche Einsparung von 
\begin{eqnarray}
366,67 \mbox{h} \cdot \eur{(23 + 12)}{\mbox{/h}} = \eur{12833,45}
\end{eqnarray}

Die Amortisationsdauer beträgt also $\frac{\eur{1015,00}}{\eur{12833,45}\mbox{/Jahr}} \approx 0,08 \mbox{ Jahre} \approx 1 \mbox{ Monat}$.


\subsection{Nutzwertanalyse}
\label{sec:Nutzwertanalyse}
\begin{itemize}
	\item Darstellung des nicht-monetären Nutzens (\zB Vorher-/Nachher-Vergleich anhand eines Wirtschaftlichkeitskoeffizienten). 
\end{itemize}

\paragraph{Beispiel}
Ein Beispiel für eine Entscheidungsmatrix findet sich in Kapitel~\ref{sec:Architekturdesign}: \nameref{sec:Architekturdesign}.


\subsection{Anwendungsfälle}
\label{sec:Anwendungsfaelle}
Der mit Abstand wichtigste Anwendungsfall ist, dass das Programm nach Angabe von Datenstrom und Schema den Datenstrom entsprechend dem Schema zergliedert anzeigt.
Der Kunde nannte noch einige Wünsche zur Funktionalität. Ein Usecase-Diagramm ist im Anhang beigefügt.


\subsection{Qualitätsanforderungen}
\label{sec:Qualitaetsanforderungen}
Schemas müssen frei angebbar sein und Datenströme richtig zergliedert werden.
Beim Abnahmetest soll dies anhand der 5 wichtigsten Schemadateien überprüft werden.
Performance ist ziemlich egal, das Programm soll aber flüssig benutzbar sein (Zergliederung von Datenstrom und Anzeige in unter 5 Sekunden).
Weil die Benutzer Profis sind, ist es nicht unbedingt notwendig, für alles eine Eingabemaske bereitzustellen. Es genügt auch, wenn eine Einstellung durch Editieren einer Konfigurations-Datei geändert werden kann.


\subsection{Lastenheft/Fachkonzept}
\label{sec:Lastenheft}
\begin{itemize}
	\item Auszüge aus dem Lastenheft/Fachkonzept, wenn es im Rahmen des Projekts erstellt wurde.
	\item Mögliche Inhalte: Funktionen des Programms (Muss/Soll/Wunsch), User Stories, Benutzerrollen
\end{itemize}

\paragraph{Beispiel}
Ein Beispiel für ein Lastenheft findet sich im \Anhang{app:Lastenheft}. 

\Zwischenstand{Analysephase}{Analyse}
