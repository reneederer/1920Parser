% !TEX root = ../Projektdokumentation.tex
\section{Analysephase} 
\label{sec:Analysephase}

\subsection{Ist-Analyse} 
\label{sec:IstAnalyse}
Die Ist-Analyse fand durch Befragung von Herrn Kemmer statt. 
Die Entwickler der Abteilung Warehouse müssen bei Kundenreklamationen, Programmänderungen und Neuentwicklungen die Werte in den beschriebenen Datenströme analysieren. Diese stammen aus den Logdateien des zentralen Lagerrechners, aus der Eingabemaske des Mainframes oder auch aus erhaltenen Email-Anhängen. Um die Bedeutung der Werte im Datenstrom herauszufinden, zählen die Entwickler gegenwärtig die Bytes in Schema und Datenstrom ab, einige erfahrenere kennen die wichtigsten Schemadateien auch zum Teil auswendig.
Im Warehouse arbeiten etwa 20 Entwickler, etwa 10 davon haben regelmäßig mit solchen Datenströmen zu tun. Es werden Dutzende verschiedene 1920Schemas für verschiedenste Zwecke verwendet, aber mit nur etwa 10 arbeiten die Entwickler regelmäßig.


\subsection{Wirtschaftlichkeitsanalyse}
\label{sec:Wirtschaftlichkeitsanalyse}
Das Programm soll den Entwicklern das bisher notwendige, fehleranfällige Abzählen von Zeichen in Schema und Datenstrom abnehmen. Das Projekt verspricht dadurch nicht nur, den Entwicklern Zeit zu sparen, sondern auch Zählfehler wirkungsvoll zu verhindern.

\subsubsection{\gqq{Make or Buy}-Entscheidung}
\label{sec:MakeOrBuyEntscheidung}
Die Anforderungen sind so speziell, dass fast auszuschließen ist, dass es außerhalb von Phoenix ein Programm gibt, das die Anforderungen erfüllt. Zu bemerken ist aber, dass viele Programme bei Phoenix mit Schemadateien arbeiten. Der Autor fragte deshalb nach, ob Phoenix eventuell schon ein Programm hat, das die Anforderungen erfüllt. Der Auftraggeber antwortete, dass die Datenströme aus so unterschiedlichen Umgebungen stammen (Unix-Lagerrechner, Mainframe, Email), dass die COBOL-Programme, die sie verarbeiten, nur eingeschränkt eingesetzt werden können. Es wurde entschieden, das Programm selbst neu zu schreiben.

\subsubsection{Projektkosten}
\label{sec:Projektkosten}
Die Kosten für die Durchführung des Projekts setzen sich sowohl aus Personal-, als auch aus Ressourcenkosten zusammen.
Der Projektersteller ist Umschüler und erhält deshalb von seinem Ausbildungsbetrieb keine Vergütung.

\begin{eqnarray}
7,7 \mbox{ h/Tag} \cdot 220 \mbox{ Tage/Jahr} = 1694 \mbox{ h/Jahr}\\
\eur{0}\mbox{/Monat} \cdot 13,3 \mbox{ Monate/Jahr} = \eur{0} \mbox{/Jahr}\\
\frac{\eur{0} \mbox{/Jahr}}{1694 \mbox{ h/Jahr}} = \eur{0,00}\mbox{/h}
\end{eqnarray}

Dadurch ergibt sich also ein Stundenlohn von \eur{0,00}
Die Durchführungszeit des Projekts beträgt 70 Stunden. Für die Nutzung von Ressourcen\footnote{Räumlichkeiten, Arbeitsplatzrechner etc.} wird 
ein pauschaler Stundensatz von \eur{12} angenommen. Für die anderen Mitarbeiter wird pauschal ein Stundenlohn von \eur{23} angenommen. 
Eine Aufstellung der Kosten befindet sich in Tabelle~\ref{tab:Kostenaufstellung} und sie betragen insgesamt \eur{1015,00}.
\tabelle{Kostenaufstellung}{tab:Kostenaufstellung}{Kostenaufstellung.tex}


\subsubsection{Amortisationsdauer}
\label{sec:Amortisationsdauer}
Es wird davon ausgegangen, dass ausschließlich Entwickler der Abteilung Lager das Programm nutzen werden. Nach Einschätzung von Herrn Kemmer arbeitet die Hälfte der 20 Entwickler regelmäßig mit Schemadateien und dass das Programm jedem täglich 10 Minuten einsparen kann.
Bei einer Einsparung von 10 Minuten am Tag für 10 Entwickler an 220 Arbeitstagen im Jahr ergibt sich eine gesamte Zeiteinsparung von 
\begin{eqnarray}
10 \cdot 220 \mbox{ Tage/Jahr} \cdot 10 \mbox{ min/Tag} = 22000 \mbox{ min/Jahr} \approx 366,67 \mbox{ h/Jahr} 
\end{eqnarray}

Dadurch ergibt sich eine jährliche Einsparung von 
\begin{eqnarray}
366,67 \mbox{h} \cdot \eur{(23 + 12)}{\mbox{/h}} = \eur{12833,45}
\end{eqnarray}

Die Amortisationsdauer beträgt also $\frac{\eur{1015,00}}{\eur{12833,45}\mbox{/Jahr}} \approx 0,08 \mbox{ Jahre} \approx 1 \mbox{ Monat}$.


\subsection{Anwendungsfälle}
\label{sec:Anwendungsfaelle}
Der mit Abstand wichtigste Anwendungsfall ist, dass das Programm nach Angabe von Datenstrom und Schema den Datenstrom entsprechend dem Schema zergliedert anzeigt.
Der Kunde nannte noch einige Wünsche mehr zur Funktionalität. Abbildung \ref{app:Usecase} stellt die Anwendungsfälle in einem Usecase-Diagramm dar, sie befindet sich im Anhang.


\subsection{Qualitätsanforderungen}
\label{sec:Qualitaetsanforderungen}
Schemas müssen frei angebbar sein und Datenströme richtig zergliedert werden.
Beim Abnahmetest soll dies anhand der 5 wichtigsten Schemadateien überprüft werden.
Die Performance des Programmes ist ziemlich egal, es soll aber flüssig benutzbar sein (Zergliederung von Datenstrom und Anzeige in unter 5 Sekunden).
Weil die Benutzer Profis sind, ist es nicht unbedingt notwendig, für alles eine Eingabemaske bereitzustellen. Es genügt auch, wenn eine Einstellung durch Editieren einer Konfigurations-Datei geändert werden kann.


\Zwischenstand{Analysephase}{Analyse}

