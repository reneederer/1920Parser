% !TEX root = ../Projektdokumentation.tex
\section{Analysephase} 
\label{sec:Analysephase}


\subsection{Ist-Analyse} 
\label{sec:IstAnalyse}



\subsection{Wirtschaftlichkeitsanalyse}
\label{sec:Wirtschaftlichkeitsanalyse}
Das Projekt verspricht nicht eine deutliche Zeitersparnis für die Warehouse-Programmierer, sondern auch eine verringerte Fehlerquote.


\subsubsection{\gqq{Make or Buy}-Entscheidung}
\label{sec:MakeOrBuyEntscheidung}
Die Anforderungen sind sehr speziell, es ist daher unwahrscheinlich, dass es ein fertiges Programm gibt, das die Anforderungen erfüllt. Das Produkt wird daher selbst erstellt.


\subsubsection{Projektkosten}
\label{sec:Projektkosten}
Im Rahmen des Projektes fallen Kosten für Entwicklung und Abnahmetest an.

\paragraph{Beispielrechnung (verkürzt)}
Die Kosten für die Durchführung des Projekts setzen sich sowohl aus Personal-, als auch aus Ressourcenkosten zusammen.
Der Projektersteller ist Umschüler und erhält seinem Ausbildungsbetrieb keine Vergütung.

\begin{eqnarray}
7,7 \mbox{ h/Tag} \cdot 220 \mbox{ Tage/Jahr} = 1694 \mbox{ h/Jahr}\\
\eur{0}\mbox{/Monat} \cdot 13,3 \mbox{ Monate/Jahr} = \eur{0} \mbox{/Jahr}\\
\frac{\eur{0} \mbox{/Jahr}}{1694 \mbox{ h/Jahr}} = \eur{0,00}\mbox{/h}
\end{eqnarray}

Dadurch ergibt sich also ein Stundenlohn von \eur{0,00}
Die Durchführungszeit des Projekts beträgt 70 Stunden. Für die Nutzung von Ressourcen\footnote{Räumlichkeiten, Arbeitsplatzrechner etc.} wird 
ein pauschaler Stundensatz von \eur{12} angenommen. Für die anderen Mitarbeiter wird pauschal ein Stundenlohn von \eur{23} angenommen. 
Eine Aufstellung der Kosten befindet sich in Tabelle~\ref{tab:Kostenaufstellung} und sie betragen insgesamt \eur{1015,00}.
\tabelle{Kostenaufstellung}{tab:Kostenaufstellung}{Kostenaufstellung.tex}


\subsubsection{Amortisationsdauer}
\label{sec:Amortisationsdauer}
\begin{itemize}
	\item Welche monetären Vorteile bietet das Projekt (\zB Einsparung von Lizenzkosten, Arbeitszeitersparnis, bessere Usability, Korrektheit)?
	\item Wann hat sich das Projekt amortisiert?
\end{itemize}

\paragraph{Beispielrechnung (verkürzt)}
Bei einer Zeiteinsparung von 5 Minuten am Tag für jeden der 20 Anwender und 220 Arbeitstagen im Jahr ergibt sich eine gesamte Zeiteinsparung von 
\begin{eqnarray}
20 \cdot 220 \mbox{ Tage/Jahr} \cdot 5 \mbox{ min/Tag} = 22000 \mbox{ min/Jahr} \approx 366,67 \mbox{ h/Jahr} 
\end{eqnarray}

Dadurch ergibt sich eine jährliche Einsparung von 
\begin{eqnarray}
366,67 \mbox{h} \cdot \eur{(23 + 12)}{\mbox{/h}} = \eur{12833,45}
\end{eqnarray}

Die Amortisationszeit beträgt also $\frac{\eur{1015,00}}{\eur{12833,45}\mbox{/Jahr}} \approx 0,08 \mbox{ Jahre} \approx 1 \mbox{ Monat}$.


\subsection{Nutzwertanalyse}
\label{sec:Nutzwertanalyse}
\begin{itemize}
	\item Darstellung des nicht-monetären Nutzens (\zB Vorher-/Nachher-Vergleich anhand eines Wirtschaftlichkeitskoeffizienten). 
\end{itemize}

\paragraph{Beispiel}
Ein Beispiel für eine Entscheidungsmatrix findet sich in Kapitel~\ref{sec:Architekturdesign}: \nameref{sec:Architekturdesign}.


\subsection{Anwendungsfälle}
\label{sec:Anwendungsfaelle}


\paragraph{Beispiel}
Ein Beispiel für ein Use Case-Diagramm findet sich im \Anhang{app:UseCase}.


\subsection{Qualitätsanforderungen}
\label{sec:Qualitaetsanforderungen}
\begin{itemize}
	\item Welche Qualitätsanforderungen werden an die Anwendung gestellt (\zB hinsichtlich Performance, Usability, Effizienz \etc (siehe \citet{ISO9126}))?
	\item Einer oder mehrere interessante (!) Anwendungsfälle könnten exemplarisch durch ein Aktivitätsdiagramm oder eine \ac{EPK} detailliert beschrieben werden. 
\end{itemize}


\subsection{Lastenheft/Fachkonzept}
\label{sec:Lastenheft}
\begin{itemize}
	\item Auszüge aus dem Lastenheft/Fachkonzept, wenn es im Rahmen des Projekts erstellt wurde.
	\item Mögliche Inhalte: Funktionen des Programms (Muss/Soll/Wunsch), User Stories, Benutzerrollen
\end{itemize}

\paragraph{Beispiel}
Ein Beispiel für ein Lastenheft findet sich im \Anhang{app:Lastenheft}. 

\Zwischenstand{Analysephase}{Analyse}
