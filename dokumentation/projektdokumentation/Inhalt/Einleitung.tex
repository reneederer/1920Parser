% !TEX root = ../Projektdokumentation.tex
\section{Einleitung}
\label{sec:Einleitung}
Ausbildungsbetrieb ist die Phoenix Group IT GmbH.

\subsection{Auftraggeber}
\label{sec:Auftraggeber}
Die Phoenix Group IT GmbH ist der IT-Dienstleister des Pharmagroßhändlers Phoenix Pharmahandel GmbH \& Co. KG. Phoenix Pharmahandel ist mit seinen Tochtergesellschaften unter dem Namen ``Phoenix Group'' europaweit tätig mit etwa 30000 Mitarbeitern. Haupttätikeiten der Phoenix Group ist die Belieferung von Apotheken mit Medikamenten.

Ausbildungsbetrieb Autors und Auftraggeber des Projektes ist die Phoenix Group IT GmbH. Sie hat etwa 200 Mitarbeiter und unterstützt Phoenix Pharmahandel durch die Bereitstellung von IT-Dienstleistungen.

\subsection{Projektumfeld} 
\label{sec:Projektumfeld}
Phoenix hat für die Datenverarbeitung im Bereich Warehouse ein eigenes Dateiformat entwickelt, im Folgenden 1920Schema genannt. 1920Schemas sind Schnittstellendateien in denen Variablen definiert werden. Anhand von 1920Schemas werden Datenströme zergliedert. 1920Schemas dienen als Vorlage für Copybooks\footnote{COBOL-Datei, in der eine Variablenstruktur definiert wird}, als Schnittstelle zu SSORT\footnote{IBM-Programm, zeigt Copybooks an} und als Schnittstelle um Daten vom Mainframe zum Lagerrechner zu schicken, und dort in Logdateien zu speichern.

1920Schemas beschreiben einen Satz hierarchisch gegliederter Variablen, und für jede Variable deren Typ und Größe in Bytes. Datenströme von typischerweise 1920 Bytes\footnote{das Terminal Window ist 24 Zeilen * 80 Spalten groß} werden anhand der Schemas zergliedert und erhalten so eine Bedeutung. Phoenix nutzt Dutzende verschiedene 1920Schemas für die Datenverarbeitung, regelmäßig arbeiten die Entwickler aber nur mit etwa 10.

\subsection{Projektziel} 
\label{sec:Projektziel}
Ziel des Projektes ist es, ein Programm (1920Parser) zu schreiben, in dem ein Datenstrom und ein 1920Schema angegeben werden, und das den Datenstrom anhand des Schemas zergliedert anzeigt.


\subsection{Projektbegründung}
\label{sec:Projektbegruendung}
Bei Kundenreklamationen, Änderungen an Programmen und Neuentwicklungen stehen die
Programmierer vor zwei Aufgaben:

\begin{itemize}
\item Wert einer Schemadatei-Variablen in einem Datenstrom finden.
\item Datenstrom-Bytes einer Schemadatei-Variablen zuordnen.
\end{itemize}

Gegenwärtig zählen die Entwickler die passende Anzahl von Bytes in Schema und Datenstrom ab, einige erfahrene Entwickler erkennen bestimmte Werte in einem Datenstrom und müssen nicht beim ersten Byte anfangen zu zählenzählen.


\subsection{Projektschnittstellen}
\label{sec:Projektschnittstellen}
Benutzer des Projektes sind die Programmierer der Phoenix Group IT GmbH, hauptsächslich die der Abteilung Warehouse.

1920Parser interagiert nicht unmittelbar mit anderen Systemen\footnote{der Anwendungsfall Schema vom Mainframe herunterladen wurde im Rahmen des Projektes nicht umgesetzt}. Vorgesehen ist, dass die Benutzer die notwendigen Angaben direkt in eine Eingabemaske hineinkopieren.

Die Projektgenehmigung und die Bereitstellung von Resourcen erfolgt durch die Ausbildende Frau Birgit Günther, die Projektbetreuung und die Abnahme des Programms durch Herrn Marco Kemmer. Herr Kemmer arbeitet in der Abteilung Warehouse als COBOL-Entwickler und wird 1920Parser auch selbst verwenden.

