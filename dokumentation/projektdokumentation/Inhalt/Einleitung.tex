% !TEX root = ../Projektdokumentation.tex
\section{Projektdefinition}
\label{sec:Einleitung}
\subsection{Auftraggeber}
\label{sec:Auftraggeber}
Die Phoenix Pharmahandel GmbH \& Co. KG. und ihre Tochtergesellschaften sind europaweit unter dem Namen ``Phoenix group'' mit etwa 30000 Mitarbeitern im Pharmagroßhandel tätig. Ihre Haupttätigkeiten sind der Großhandel mit Pharmaprodukten und die Belieferung von Apotheken mit Medikamenten.

Auftraggeber des Projektes ist die Phoenix group IT GmbH. Sie hat etwa 200 Mitarbeiter und unterstützt die Phoenix group durch die Bereitstellung von IT-Dienstleistungen.

\subsection{Projektumfeld} 
\label{sec:Projektumfeld}
Die Phoenix group IT ist weiter unterteilt in die Abteilungen Inbound, Outbound und Warehouse. Das Projekt findet in der Abteilung Warehouse statt, in der überwiegend mit COBOL gearbeitet wird.

Phoenix verwendet ein firmeneigenes Dateiformat als Schnittstelle zu verschiedenen Programmen und Diensten. Es wird im Folgenden 1920Schema genannt und spielt eine zentrale Rolle für das Projekt.

\subsubsection{1920Schemas}
1920Schemas sind Textdateien, die einen Satz hierarchisch gegliederter Variablen beschreiben. Phoenix nutzt diese Schemas unter anderem als Schnittstelle, um Daten vom Mainframe zum Lagerrechner zu schicken, als Vorlage für Copybooks\footnote{COBOL-Datei, in der eine Variablenstruktur definiert wird} und als Schnittstelle zu SSORT\footnote{IBM-Programm, zeigt Copybooks an}.

\textbf{Beispiel für ein 1920Schema}
\tabelleLinks{Beispiel für ein 1920Schema}{tab:Beispiel für ein 1920Schema}{BeispielSchema.tex}

\subsubsection{Datenströme}
Anhand von 1920Schemas werden Datenströme zergliedert und erhalten dadurch eine Bedeutung. Datenströme können prinzipiell beliebige Zeichen\footnote{Ein Byte entspricht einem Zeichen. Die Angaben in 1920Schemas haben die Einheit Byte, im Zusammenhang mit dem Datenstrom ist der Begriff Zeichen aber manchmal anschaulicher. Eine Unterscheidung der beiden Begriffe ist für das Projekt unwesentlich.} enthalten, nur die Anzahl muss ausreichen, um jeder Schema-Variablen einen Wert zuzuweisen. Das Schema aus Tabelle 1 beschreibt einen 17 Byte langen Datenstrom (5+4+2*4), der so aussehen könnte:

\textbf{Beispiel für einen Datenstrom}

VN\textasciitilde \textasciitilde \textasciitilde NN\textasciitilde\textasciitilde12345678

Datenströme sind fast immer 1920 Bytes lang.


\subsection{Projektziel}
\label{sec:Projektziel}
Es soll ein Programm geschrieben werden, das nach Eingabe von Datenstrom und 1920Schema den Datenstrom dem Schema entsprechend zergliedert und anzeigt. Das Programm wird im Folgenden 1920Parser genannt.

\textbf{Gewünschte Ausgabe} (mit den Beispielwerten der Abschnitte 1.2.1 und 1.2.2)

\begin{tabularx}{0px}{l}
01 Daten\\
\ \ \ 03 Personendaten\\
\ \ \ \ \ \ 05 Vorname=VN\textasciitilde \textasciitilde \textasciitilde\\
\ \ \ \ \ \ 05 Nachname=NN\textasciitilde \textasciitilde\\
\ R03 Gesamter-Name=VN\textasciitilde \textasciitilde \textasciitilde NN\textasciitilde \textasciitilde\\
\ \ \ 03 Bestellungen(1)\\
\ \ \ \ \ \ 05 Artikelnr=1234\\
\ \ \ 03 Bestellungen(2)\\
\ \ \ \ \ \ 05 Artikelnr=5678\\
\end{tabularx}

\subsection{Projektbegründung}
\label{sec:Projektbegruendung}
Bei Kundenreklamationen, Änderungen an Programmen und Neuentwicklungen stehen die Warehouse-Programmierer oft vor den Problemen:

\begin{itemize}
\item Wert einer Schemadatei-Variablen in einem Datenstrom finden.
\item Datenstrom-Bytes einer Schemadatei-Variablen zuordnen.
\end{itemize}

Gegenwärtig zählen die Programmierer die passende Anzahl von Bytes in Schema und Datenstrom ab, einige erfahrene kennen die wichtigsten Schemadateien auch teilweise auswendig. 


\subsection{Projektschnittstellen}
\label{sec:Projektschnittstellen}
Benutzer des Programms sind die Programmierer der Phoenix Group IT GmbH, hauptsächlich die aus der Abteilung Warehouse.

1920Parser soll nicht unmittelbar mit anderen Systemen interagieren. Vorgesehen ist, dass die Benutzer die notwendigen Angaben in eine Eingabemaske hineinkopieren.

Projektgenehmigung und die Bereitstellung von Ressourcen erfolgt durch die Ausbildende Frau Birgit Günther, die Projektbetreuung und die Abnahme des Programms durch Herrn Marco Kemmer. Herr Kemmer arbeitet in der Abteilung Warehouse als COBOL-Entwickler, er kommuniziert die Kundenwünsche und will das Programm in Zukunft auch selbst nutzen.




