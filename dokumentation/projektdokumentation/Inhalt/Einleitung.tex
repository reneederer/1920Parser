% !TEX root = ../Projektdokumentation.tex
\section{Einleitung}
\label{sec:Einleitung}
\subsection{Ausbildungsbetrieb}
\label{sec:Ausbildungsbetrieb}
Die Phoenix group IT GmbH ist der IT-Dienstleister des Pharmagroßhändlers Phoenix Pharmahandel GmbH \& Co. KG. Phoenix Pharmahandel ist mit seinen Tochtergesellschaften unter dem Namen ``Phoenix group'' europaweit tätig mit etwa 30000 Mitarbeitern. Haupttätikeit der Phoenix group ist die Belieferung von Apotheken mit Medikamenten.

Ausbildungsbetrieb des Autors und Auftraggeber des Projektes ist die Phoenix group IT GmbH. Sie hat etwa 200 Mitarbeiter und unterstützt Phoenix Pharmahandel durch die Bereitstellung von IT-Dienstleistungen.

\subsection{Projektumfeld} 
\label{sec:Projektumfeld}
Die Phoenix group IT ist weiter unterteilt in die Abteilungen Inbound, Outbound und Warehouse. Das Projekt findet in der Abteilung Warehouse statt, in der überwiegend mit COBOL gearbeitet wird.
\subsubsection{1920Schemas}
Die Abteilung Warehouse verwendet ein firmeneigenes Dateiformat als Schnittstelle zu verschiedenen Programmen und Diensten (im Folgenden 1920Schema genannt).
1920Schemas sind Textdateien, die einen Satz hierarchisch gegliederter Variablen beschreiben. Phoenix verwendet diese Schemas unter anderem als Schnittstelle, um Daten vom Mainframe zum Lagerrechner zu schicken, als Vorlage für Copybooks\footnote{COBOL-Datei, in der eine Variablenstruktur definiert wird} und als Schnittstelle zu SSORT\footnote{IBM-Programm, zeigt Copybooks an}.

\textbf{Beispiel für ein 1920Schema}
\tabelleLinks{Beispiel für ein 1920Schema}{tab:Beispiel für ein 1920Schema}{BeispielSchema.tex}

\subsubsection{Datenströme}
Anhand von 1920Schemas werden Datenströme zergliedert, die dadurch eine Bedeutung erhalten. Datenströme können prinzipiell beliebige Zeichen enthalten, nur die Anzahl muss mit dem Schema zusammenpassen. Das Schema aus Tabelle 1 beschreibt einen Datenstrom mit 17 Zeichen (5+4+2*4), der so aussehen könnte:

\textbf{Beispiel für einen Datenstrom}

VN\textasciitilde \textasciitilde \textasciitilde NN\textasciitilde\textasciitilde12345678


\subsection{Projektziel}
\label{sec:Projektziel}
Ziel des Projektes ist es, ein Programm (1920Parser) zu schreiben, in dem ein Datenstrom und ein 1920Schema angegeben werden, und das den Datenstrom anhand des Schemas zergliedert anzeigt.

\textbf{Gewünschte Ausgabe} (mit den Beispielwerten der Abschnitte 1.2.1 und 1.2.2)

\begin{tabularx}{0px}{l}
01 Daten\\
\ \ \ 03 Personendaten\\
\ \ \ \ \ \ 05 Vorname=VN\textasciitilde \textasciitilde \textasciitilde\\
\ \ \ \ \ \ 05 Nachname=NN\textasciitilde \textasciitilde\\
\ R03 Gesamter-Name=VN\textasciitilde \textasciitilde \textasciitilde NN\textasciitilde \textasciitilde\\
\ \ \ 03 Bestellungen(1)\\
\ \ \ \ \ \ 05 Artikelnr=1234\\
\ \ \ 03 Bestellungen(2)\\
\ \ \ \ \ \ 05 Artikelnr=5678\\
\end{tabularx}

\subsection{Projektbegründung}
\label{sec:Projektbegruendung}
Bei Kundenreklamationen, Änderungen an Programmen und Neuentwicklungen stehen die Warehouse-Programmierer oft vor den Aufgaben:

\begin{itemize}
\item Wert einer Schemadatei-Variablen in einem Datenstrom finden.
\item Datenstrom-Bytes einer Schemadatei-Variablen zuordnen.
\end{itemize}

Gegenwärtig zählen die Programmierer die passende Anzahl von Bytes in Schema und Datenstrom ab, einige erfahrene kennen die wichtigsten Schemadateien auch teilweise auswendig. 


\subsection{Projektschnittstellen}
\label{sec:Projektschnittstellen}
Benutzer des Programms sind die Programmierer der Phoenix Group IT GmbH, hauptsächlich aus der Abteilung Warehouse.
1920Parser soll nicht unmittelbar mit anderen Systemen interagieren. Vorgesehen ist, dass die Benutzer die notwendigen Angaben direkt in eine Eingabemaske eingeben, bzw. hineinkopieren.
Die Projektgenehmigung und die Bereitstellung von Resourcen erfolgt durch die Ausbildende Frau Birgit Günther, die Projektbetreuung und die Abnahme des Programms durch Herrn Marco Kemmer. Herr Kemmer arbeitet in der Abteilung Warehouse als COBOL-Entwickler und will das Programm auch selbst nutzen.

