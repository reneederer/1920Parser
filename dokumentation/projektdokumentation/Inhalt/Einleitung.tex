% !TEX root = ../Projektdokumentation.tex
\section{Einleitung}
\label{sec:Einleitung}


\subsection{Auftraggeber}
\label{sec:Auftraggeber}
Die Phoenix Group IT GmbH ist der IT-Dienstleister des Pharmagroßhändlers Phoenix Pharmahandel
GmbH \& Co. KG. Phoenix Pharmahandel ist unter dem Namen ``Phoenix Group'' europaweit tätig mit
etwa 30000 Mitarbeitern. Haupttätikeiten sind die Bestellannahme von Apotheken und die Belieferung
von Apotheken mit Medikamenten. Ausbildungsbetrieb und Auftraggeber des Projektes ist die Phoenix Group IT. Sie hat etwa 200 Mitarbeiter und unterstützt Phoenix Pharmahandel durch die Bereitstellung von Infrastruktur wie z. B. Servern, außerdem schreibt sie Programme. Die Phoenix Group IT gliedert sich in die drei Bereiche Inbound, Warehouse und Outbound. Rechnungsbearbeitung, Bestellannahme, Reklamations-verarbeitung, Auswahl des besten Zulieferers, Zeitplanung, Lagerplanung, Mitarbeiterplanung. Im 

\subsection{Projektumfeld} 
\label{sec:Projektumfeld}
Für die Datenverarbeitung im Bereich Warehouse kommt COBOL zum Einsatz. Phoenix hat  ein eigenes Dateiformat (im Folgenden 1920Schema genannt) entwickelt, das zum einen als Copybook-Vorlage\footnote{COBOL-Datei, in der eine Variablenstruktur definiert wird} dient, und zum anderen als Schnittstelle, um Datenströme vom Mainframe zum Lagerrechner zu schicken und dort in Logdateien zu speichern.

1920Schemas beschreiben einen Satz hierarchisch gegliederter Variablen, und für jede Variable deren Typ und Größe in Bytes. Datenströme von 1920 Bytes\footnote{1920 weil das Terminal Window 24 Zeilen * 80 Spalten groß ist} werden anhand der Schemas zergliedert und erhalten so eine Bedeutung. Phoenix verwendet Dutzende verschiedene 1920Schemas für die Datenverarbeitung, regelmäßig arbeiten die Entwickler aber nur mit etwa 10.

\subsection{Projektziel} 
\label{sec:Projektziel}
Ziel des Projektes ist es, ein Programm zu schreiben, in dem ein Datenstrom und ein 1920Schema angegeben werden, und das den Datenstrom anhand des Schemas zergliedert anzeigt.


\subsection{Projektbegründung}
\label{sec:Projektbegruendung}
Bei Kundenreklamationen, Änderungen an Programmen und Neuentwicklungen stehen die
Programmierer Phoenix vor zwei wiederkehrenden Aufgaben:

\begin{itemize}
\item Wert einer Schemadatei-Variablen in einem Datenstrom finden.
\item Datenstrom-Bytes einer Schemadatei-Variablen zuordnen.
\end{itemize}

Gegenwärtig zählen viele Entwickler die passende Anzahl von Bytes in Schema und Datenstrom ab, einige erfahrene kennen die wichtigsten Schemadateien auch zum Teil auswendig.



\subsection{Projektschnittstellen}
\label{sec:Projektschnittstellen}
Benutzer des Projektes sind die Programmierer der Phoenix Group IT.
1920Parser interagiert nicht unmittelbar mit anderen Systemen. Vorgesehen ist, dass die benötigten Angaben direkt in eine Eingabemaske hineinkopiert werden. Im Betrieb ist auf den Entwickler-Rechnern Windows 7 im Einsatz, und unter diesem Betriebssystem soll auch 1920Parser laufen.
Die Projektgenehmigung und die Bereitstellung von Resourcen erfolgt durch die Ausbildende Frau Birgit Günther, die Abnahme durch den Projektbetreuer Herrn Marco Kemmer.

