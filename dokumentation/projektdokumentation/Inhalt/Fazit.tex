% !TEX root = ../Projektdokumentation.tex
\section{Fazit} 
\label{sec:Fazit}

\subsection{Soll-/Ist-Vergleich}
\label{sec:SollIstVergleich}

\begin{itemize}
	\item Wurde das Projektziel erreicht und wenn nein, warum nicht?
	\item Ist der Auftraggeber mit dem Projektergebnis zufrieden und wenn nein, warum nicht?
	\item Wurde die Projektplanung (Zeit, Kosten, Personal, Sachmittel) eingehalten oder haben sich Abweichungen ergeben und wenn ja, warum?
	\item Hinweis: Die Projektplanung muss nicht strikt eingehalten werden. Vielmehr sind Abweichungen sogar als normal anzusehen. Sie müssen nur vernünftig begründet werden (\zB durch Änderungen an den Anforderungen, unter-/überschätzter Aufwand).
\end{itemize}

\paragraph{Beispiel (verkürzt)}
Wie in Tabelle~\ref{tab:Vergleich} zu erkennen ist, konnte die Zeitplanung bis auf wenige Ausnahmen eingehalten werden.
\tabelle{Soll-/Ist-Vergleich}{tab:Vergleich}{Zeitnachher.tex}


\subsection{Lessons Learned}
\label{sec:LessonsLearned}
Rekursion ist nicht gut darstellbar mit UML. Für Polymorphie gilt das selbe. Stacks sind super. Rekursion vereinfacht manche Aufgaben enorm. Bemerkenswert, wie sehr sich die Anforderungen ausgeweitet haben.


\subsection{Ausblick}
\label{sec:Ausblick}

