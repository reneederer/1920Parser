% !TEX root = ../Projektdokumentation.tex
\section{Fazit} 
\label{sec:Fazit}

\subsection{Soll-/Ist-Vergleich}
\label{sec:SollIstVergleich}
Die Musskriterien wurden vollständig umgesetzt. Die Projektvorgaben wurden vollständig umgesetzt.
Zusätzlich wurde Funktionalität implementiert zum Speichern von Schemas und zur automatisierten Auswahl des zum Datenstrom passenden Schemas.
Für den Anwendungsfall ''Datenstrom importieren`` wurde ein Bedienkonzept entwickelt, aber mit der Umsetzung nicht begonnen. 
Ausblenden von Schemaknoten ist nicht implementiert.


\begin{itemize}
	\item Wurde das Projektziel erreicht und wenn nein, warum nicht?
	\item Ist der Auftraggeber mit dem Projektergebnis zufrieden und wenn nein, warum nicht?
	\item Wurde die Projektplanung (Zeit, Kosten, Personal, Sachmittel) eingehalten oder haben sich Abweichungen ergeben und wenn ja, warum?
	\item Hinweis: Die Projektplanung muss nicht strikt eingehalten werden. Vielmehr sind Abweichungen sogar als normal anzusehen. Sie müssen nur vernünftig begründet werden (\zB durch Änderungen an den Anforderungen, unter-/überschätzter Aufwand).
\end{itemize}

\paragraph{Beispiel (verkürzt)}
Wie in Tabelle~\ref{tab:Vergleich} zu erkennen ist, konnte die Zeitplanung bis auf wenige Ausnahmen eingehalten werden.
\tabelle{Soll-/Ist-Vergleich}{tab:Vergleich}{Zeitnachher.tex}


\subsection{Lessons Learned}
\label{sec:LessonsLearned}
RichTextBox ist besser als TextBox.
Interessant war, wie sich der Projektumfang erweitert hatte (ursprünglich war der Vorschlag, dass nur ein bestimmtes Schema zergliedert werden soll)
Rekursion ist nicht gut darstellbar mit UML. Für Polymorphie gilt das selbe. Stacks sind super. Rekursion vereinfacht manche Aufgaben enorm. Bemerkenswert, wie sehr sich die Anforderungen ausgeweitet haben.
Bemerkenswert war, dass sich der Projektumfang stark erhöhte (ursprünglich wurde mir der Projektvorschlag gemacht, nur die Schemadatei VK60 zu zergliedern).



\subsection{Ausblick}
\label{sec:Ausblick}
Im Rahmen des Projektes wurden nicht alle Wunschkriterien erfüllt. Der Autor hofft aber, dass diese Aufgabe vielleicht an nachfolgende Auszubildende übergeben wird und sich 1920Parser noch weiterentwickelt.

Einige Ideen für Weiterentwicklungen von 1920Parser sind:
\begin{itemize}
\item Ausblenden von angebbaren Knotenpunkten
\item Implementierung des Anwendungsfalls ''Datenstrom importieren``
\item Bei Klick auf eine Zeile im Schema wird zur entsprechenden Zeile im Ergebnis-Textfeld gesprungen.
\item Bei Klick ins Datenstrom-Textfeld wird zur entsprechenden Zeile im Ergebnis-Datenfeld gesprungen
\item Erweiterung der Eingabemaske für Schema speichern, so dass mehrere Angaben zur automatischen Auswahl getroffen werden können.
\item Automatisches Herunterladen von Schemas vom Mainframe
\item 
\end{itemize}
1920Parser könnte eine nette Spielwiese für einen zukünftigen Auszubildenden werden.






