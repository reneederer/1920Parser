% !TEX root = ../Projektdokumentation.tex
\section{Fazit} 
\label{sec:Fazit}

\subsection{Soll-/Ist-Vergleich}
\label{sec:SollIstVergleich}
Wie gewünscht können im Programm Datenstrom und Schema frei angegeben werden. Das Ergebnis wurde bei allen Tests richtig angezeigt.

Zur Zeitplanung ist zu sagen, dass insbesondere das Zuweisen der Werte aus dem Datenstrom und die Ausgabe des Baumes als String deutlich weniger Zeit in Anspruch nahmen, als ursprünglich geplant. So konnte noch der Anwendungsfall ``Schema speichern'' im Rahmen des Projektes umgesetzt werden.
Herr Kemmer ist mit dem Programm sehr zufrieden.
\tabelle{Soll-/Ist-Vergleich}{tab:Vergleich}{Zeitnachher.tex}
\subsection{Lessons Learned}
\label{sec:LessonsLearned}
Interessant war, wie sich die Anforderungen erweiterten (ursprünglich sollte nur ein bestimmtes Schema zergliedert werden). Unit-Tests zu schreiben hat sich ausgezahlt, einmal durch höhere Chancen, dass das Programm tut, was es soll, aber auch dadurch, dass es vermutlich Debugging gespart hat. Der Autor konnte durch das Projekt sein Schulwissen vertiefen und wichtige Erfahrungen für die Planung und Umsetzung größerer Projekte sammeln. Er ist ein noch größerer Fan von Stacks als vorher.
\subsection{Ausblick}
\label{sec:Ausblick}
Im Rahmen des Projektes wurden nicht alle Wunschkriterien erfüllt. Der Autor hofft aber, dass diese Aufgabe an einen nachfolgenden Auszubildenden übergeben wird und sich 1920Parser noch weiterentwickelt. Phoenix erwägt seit Jahren, seine COBOL-Programme zu portieren. Dann könnte vielleicht der Algorithmus von 1920Parser für eine Teilaufgabe verwendet werden.





