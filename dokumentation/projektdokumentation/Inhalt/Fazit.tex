% !TEX root = ../Projektdokumentation.tex
\section{Fazit} 
\label{sec:Fazit}

\subsection{Soll-/Ist-Vergleich}
\label{sec:SollIstVergleich}
Wie gewünscht können im Programm Datenstrom und Schema frei angegeben werden. Das Ergebnis wird richtig angezeigt.
Insbesondere das Zuweisen der Werte aus dem Datenstrom und die Ausgabe des Baumes als String nahmen deutlich weniger Zeit in Anspruch, als ursprünglich geplant. So konnte noch der Anwendungsfall ``Schema speichern'' im Rahmen des Projektes umgesetzt werden.
Herr Kemmer ist mit dem Programm sehr zufrieden.
\tabelle{Soll-/Ist-Vergleich}{tab:Vergleich}{Zeitnachher.tex}
\subsection{Lessons Learned}
\label{sec:LessonsLearned}
RichTextBox ist besser als TextBox.
Interessant war, wie sich der Projektumfang erweitert hatte (ursprünglich war der Vorschlag, dass nur ein bestimmtes Schema zergliedert werden soll)
Rekursion ist nicht gut darstellbar mit UML. Für Polymorphie gilt das selbe. Stacks sind super. Rekursion vereinfacht manche Aufgaben enorm. Bemerkenswert, wie sehr sich die Anforderungen ausgeweitet haben.
Bemerkenswert war, dass sich der Projektumfang stark erhöhte (ursprünglich wurde mir der Projektvorschlag gemacht, nur die Schemadatei VK60 zu zergliedern).
\subsection{Ausblick}
\label{sec:Ausblick}
Im Rahmen des Projektes wurden nicht alle Wunschkriterien erfüllt. Der Autor hofft aber, dass diese Aufgabe vielleicht an nachfolgende Auszubildende übergeben wird und sich 1920Parser noch weiterentwickelt. Vielleicht wird einmal die Größe der Datenströme geändert, dann könnte der Algorithmus von 1920Parser verwendet werden.





