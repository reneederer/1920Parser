%!TEX root = Projektdokumentation.tex
%\section{Implementierung - Schema speichern}
%\label{sec:ImplementierungSchemaSpeicherun}
%Die Implementierung dieses Anwendungsfalls war eher Routine.
%\subsection{Erstellung des XSD-Datei}
%\label{sec:XSDDATEIERstellung}
%Es wurde eine xml-Datei (schemaConfig.xml) mit den benötigten Angaben und der %gewünschten Struktur erstellt. Mit xsd.exe wurde daraus eine XSD-Datei erzeugt, %und mit erneutem Aufruf von xsd.exe aus dieser die C\#-Klasse SchemaConfig.
%Diese wird verwendet um die der Xml-Datei zu deserialisieren.



\section{Implementierungsphase - Schemas speichern und gespeicherte Schemas auswählen}
\label{sec:Implementierungsphase23e}
Da die Funktionalität zum Lesen der ``schemas''-Ordners in keine der vorhandenen Klassen passte, wurde die neue Klasse SchemaManager erstellt.

Es wurde eine Combobox zur GUI hinzugefügt. Der Konstruktor von 1920ParserWindow wurde so angepasst, dass er die Methode getSchemas() der Klasse SchemaManager aufruft und die Combobox die zurückgegebenen Dateinamen als Einträge erhält.

Zur Auswahl eines Schemas wurde über den \ac{VS} GUI-Designer die Methode cmbSchemas\_SelectedIndexChanged() erstellt, die die Methode getFileContent() Klasse SchemaManager bei jeder Combobox-Indexänderung aufruft und den Dateiinhalt als string zurückgegeben bekommt. Das Schema-Textfeld zeigt diesen string an.

\Zwischenstand{Implementierung des Anwendungsfalls Schemas speichern}{Implementierung1}


