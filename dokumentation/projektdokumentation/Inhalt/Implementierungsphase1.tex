% !TEX root = Projektdokumentation.tex
\section{Implementierung - Schema speichern}
\label{sec:ImplementierungSchemaSpeicherun}
Die Implementierung dieses Anwendungsfalls war eher Routine.
\subsection{Erstellung des XSD-Datei}
\label{sec:XSDDATEIERstellung}
Es wurde eine xml-Datei (schemaConfig.xml) mit den benötigten Angaben und der gewünschten Struktur erstellt. Mit xsd.exe wurde daraus eine XSD-Datei erzeugt, und mit erneutem Aufruf von xsd.exe aus dieser die C\#-Klasse SchemaConfig.
Diese wird verwendet um die der Xml-Datei zu deserialisieren.


