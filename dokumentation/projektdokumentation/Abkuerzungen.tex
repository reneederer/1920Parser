% !TEX root = Projektdokumentation.tex

% Es werden nur die Abkürzungen aufgelistet, die mit \ac definiert und auch benutzt wurden. 
%
% \acro{VERSIS}{Versicherungsinformationssystem\acroextra{ (Bestandsführungssystem)}}
% Ergibt in der Liste: VERSIS Versicherungsinformationssystem (Bestandsführungssystem)
% Im Text aber: \ac{VERSIS} -> Versicherungsinformationssystem (VERSIS)

% Hinweis: allgemein bekannte Abkürzungen wie z.B. bzw. u.a. müssen nicht ins Abkürzungsverzeichnis aufgenommen werden
% Hinweis: allgemein bekannte IT-Begriffe wie Datenbank oder Programmiersprache müssen nicht erläutert werden,
%          aber ggfs. Fachbegriffe aus der Domäne des Prüflings (z.B. Versicherung)

% Die Option (in den eckigen Klammern) enthält das längste Label oder
% einen Platzhalter der die Breite der linken Spalte bestimmt.
\begin{acronym}[WWWWW]
	\acro{COBOL}{Common Business-Oriented Language}
	\acro{GUI}{Graphical User Interface}
	\acro{IDE}{Integrated Development Environment}
	\acro{Linq}{Language Integrated Queries}
	\acro{Regex}{Regular Expression}
	\acro{SVN}{Subversion}
	\acro{TDD}{Test Driven Development}
	\acro{UML}{Unified Modeling Language}
	\acro{WPF}{Windows Presentation Foundation}
    \acro{VS}{Visual Studio}
\end{acronym}
